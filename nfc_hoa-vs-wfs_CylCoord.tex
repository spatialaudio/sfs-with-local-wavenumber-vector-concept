\documentclass[a4paper,BCOR=15mm,10pt,twoside]{scrartcl}
\usepackage[utf8]{inputenc}       
\usepackage[T1]{fontenc}
\usepackage{lmodern}
\KOMAoptions{DIV=14}
\usepackage[english]{babel}
\usepackage{xcolor}
\usepackage{url}
\usepackage{amsmath}
\usepackage{comment}
\newcommand{\fscom}[2][red]{\textcolor{#1}{#2}}  %for comments in red
\newcommand\e{\mathrm{e}}  % euler
\newcommand\im{\mathrm{j}}  % imaginary unit
\newcommand\fsd{\mathrm{d}}  % der/int operator
\newcommand\wc{\frac{\omega}{c}}  % omega / c
\newcommand\jwc{\im\,\frac{\omega}{c}}  % j omega / c
\newcommand\azx{\alpha}  % azimuth [0,2pi]
\newcommand\elx{\vartheta}  % colatitude [0,pi]...[north, south] pole
\renewcommand{\vec}[1]{\mathbf{#1}}  % vector = bold
\newcommand\unitn{\vec{n}}  % unit inward normal
\newcommand{\norm}[1]{|#1|}  %magnitude (length) of vector

%\excludecomment{calc}
\includecomment{calc}
\definecolor{CalcColor}{rgb}{0.0,0.0,0.5}
\newcommand{\ExCalcCol}[2][CalcColor]{\textcolor{#1}{#2}} %ausführliche Rechnungen

\title{Comparing the Implicit (WFS) and the Explicit (NFC-HOA) Solution of the 
2.5D Single Layer Potential (SLP) in Cylindrical Coordinates}
\author{Frank Schultz}
%
\begin{document}
\maketitle
%
This is a recalculation and double check of Fiete Winter's \url{nfc_hoa-vs-wfs.tex} of commit \url{7f2221a} in \url{https://github.com/spatialaudio/sfs-with-local-wavenumber-vector-concept}.

For brevity we omit the angular frequency $\omega$ dependence in the (local) wave vector $\vec{k}$, sound field phase $\phi$, sound field $S(\vec{x})$ and so on. The vector scalar product notation is $\vec{a} \cdot \vec{b}$.
%
We use $\e^{+\im \omega t}$ time convention with imaginary unit denoted by $\im$.
%
Then (local) wave vectors indicate propagating direction with convention $\vec{k} = - \nabla \phi$ using spatial gradient operator $\nabla$. Speed of sound is denoted by $c$, assumed to be constant for linear acoustics.

The first SPA w.r.t. height is given very detailed to recheck the unified 2.5D WFS framework in tutorial style.

\section{High Frequency Approximation of Helmholtz Integral Equation in Cylindrical Coordinates, WFS}

\subsection{Synthesis Integral and Approximation}

A position vector on the SSD is described by (cartesian, cylindrical, spherical) vector
\begin{align}
\vec{x}_0 = (x_0,y_0,z_0)^\mathrm{T} = (r_0,\azx_0, z_0)^\mathrm{T} = (r_0,\azx_0, \elx_0)^\mathrm{T}
\end{align}
with azimuth $0\leq \azx < 2\pi$, colatitude/zenith angle $0\leq \elx \leq \pi$, height $-\infty \leq z \leq \infty$, radius $0 < r \leq \infty$.
Here we utilize a infinite long cylindrical SSD with finite radius $R$ and for convenience cylindrical coordinates.
A position vector within this cylinder (i.e. listening space with $r<R$) is denoted by $\vec{x}$.
%
For the Helmholtz Integral Equation (HIE) the normal derivative w.r.t. inward normal $\unitn_0$ is required.
It reads
\begin{align}
\vec{n}_0 \cdot \nabla = -\frac{\partial}{\partial r} \qquad \vec{n}_0 \cdot \nabla(\cdot)\big|_{r=r_0} = -\frac{\partial}{\partial r}(\cdot)\bigg|_{r=r_0}.
\end{align}
%
The HIE for a cylindrical SSD with radius $R$ \fscom{$r_0$ as variable vs R const} is then given as
\begin{align}
S(\vec{x}) = \int\limits_{0}^{2\pi}\int\limits_{-\infty}^{+\infty}
\left[
G(\vec{x}-\vec{x}_0) \frac{\partial S(\vec{x})}{\partial r}\bigg|_{r=r_0} 
-S(\vec{x_0}) \frac{\partial G(\vec{x}-\vec{x}_0)}{\partial r_0}\cdot \unitn(\vec{x}_0)
\right]
\fsd z_0 \, R \, \fsd \azx_0.
\end{align}
%
We utilize the high-frequency approximation of the HIE (cf. \cite[(57),(58)]{Firtha2018}, \cite[(3.42)]{Firtha2018Diss}, \cite[(2.36)]{Schultz2016Diss})
\begin{align}
\label{eq:HIE_FAR1}
S(\vec{x}) \approx \int\limits_{0}^{2\pi}\int\limits_{-\infty}^{+\infty}
\im 
\left[
k_{S,r}(\vec{x_0}) + k_{G,r}(\vec{x} - \vec{x_0})
\right]
S(\vec{x_0})\,G(\vec{x}-\vec{x}_0)
\fsd z_0 \, R \, \fsd \azx_0.
\end{align}
%
Using the point source for the target/primary sound field $P(\vec{x}) = \frac{\e^{-\jwc |\vec{x}-\vec{x}_s|}}{4 \pi |\vec{x}-\vec{x}_s|}$ originating from $\vec{x}_S$ as well as the 3D Green's freefield function $G(\vec{x}-\vec{x}_0) = \frac{\e^{-\jwc |\vec{x}-\vec{x}_0|}}{4 \pi |\vec{x}-\vec{x}_0|}$ at $\vec{x}_0$, we get (cf. \cite[Sec. 2.2.1]{Schultz2016Diss})
\begin{align}
&k_{S,r}(\vec{x_0}) = \wc \cos\azx_S\qquad
&k_{G,r}(\vec{x_0}) = \wc \cos\azx_G\\
&\cos\azx_S = \frac{\vec{x_0}-\vec{x}_S}{|\vec{x_0}-\vec{x}_S|} \cdot \vec{n}(\vec{x}_0)\qquad
&\cos\azx_G = \frac{\vec{x}-\vec{x}_0}{\norm{\vec{x}-\vec{x}_0}} \cdot {\unitn(\vec{x}_0)}.
\end{align}
%
More generally by splitting the scalar product and introducing the local wave vector concept (cf. \cite[Sec. 3.1.1]{Firtha2018Diss}) of arbitrary sound fields we get
\begin{align}
\label{eq:kSrkGr}
k_{S,r}(\vec{x_0}) =& \vec{k}_S(\vec{x}_0) \cdot \unitn(\vec{x}_0) = - \nabla_{\vec{x}_0}\phi_S(\vec{x}_0) \cdot \unitn(\vec{x}_0)\\
k_{G,r}(\vec{x_0}) =& \vec{k}_G(\vec{x}-\vec{x}_0) \cdot \unitn(\vec{x}_0) = -\nabla_\vec{x}\phi_G(\vec{x}-\vec{x}_0) \cdot \unitn(\vec{x}_0) = + \nabla_{\vec{x}_0} \phi_G(\vec{x}_0-\vec{x}) \cdot \unitn(\vec{x}_0)\\
k_{G,r}(\vec{x_0}) =& \vec{k}_G(\vec{x}-\vec{x}_0) \cdot \unitn(\vec{x}_0) = -\nabla_\vec{x}\phi_G(\vec{x}_0-\vec{x}) \cdot \unitn(\vec{x}_0) = + \nabla_{\vec{x}_0} \phi_G(\vec{x}-\vec{x}_0) \cdot \unitn(\vec{x}_0).
\end{align}
Note the invariance of $\vec{x}_0-\vec{x}$ and $\vec{x}-\vec{x}_0$ in the argument of Green's function for pure evaluation, but the difference of the derivative w.r.t. $\vec{x}_0$ and $\vec{x}$. So be careful when deriving phase terms.
It thus might be worth noting that
\begin{align}
\label{eq:PntSrcDerivEqualities}
(\frac{\partial \phi(\vec{x}-\vec{x}_0)}{\partial z_0} = & \frac{\partial \phi(\vec{x}_0-\vec{x})}{\partial z_0}) \neq
(\frac{\partial \phi(\vec{x}-\vec{x}_0)}{\partial z} = \frac{\partial \phi(\vec{x}_0-\vec{x})}{\partial z})
\\
\frac{\partial^2 \phi(\vec{x}-\vec{x}_0)}{\partial z_0^2} =& \frac{\partial^2 \phi(\vec{x}_0-\vec{x})}{\partial z_0^2} = \frac{\partial^2 \phi(\vec{x}-\vec{x}_0)}{\partial z^2} = \frac{\partial^2 \phi(\vec{x}_0-\vec{x})}{\partial z^2}\nonumber
\end{align} 
for the point source.
%

By splitting the involved sound fields into polar form
\begin{align}
S(\vec{x}) =& A_S(\vec{x}) \, \e^{\im \phi_S(\vec{x})}\\
G(\vec{x}-\vec{x}_0) =& A_G(\vec{x}-\vec{x}_0) \, \e^{\im \phi_G(\vec{x}-\vec{x}_0)}
\end{align}
%
we have access to the complete phase term
%
\begin{align}
\phi(\vec{x}, \vec{x_0}) = \phi_{S}(\vec{x})  + \phi_{G}(\vec{x}-\vec{x}_0)
\end{align}
%
 of the integral \eqref{eq:HIE_FAR1}, for which the stationary phase point $\vec{x}_0^*$ is to be found in the following.

\subsection{Stationary Phase Approximation w.r.t. Height}
In order to reduce the dimensionality of \eqref{eq:HIE_FAR1}, a first stationary phase approximation (SPA) w.r.t. $z_0$ is performed.
This then reduces the infinite SSD cylinder to a circle, here in $xy$-plane.
For that we need the derivatives and conditions of the aforementioned phase term as
%
\begin{align}
\frac{\partial \phi(\vec{x}, \vec{x_0})}{\partial z_0}\bigg|_{\vec{x}_0=\vec{x}_0^*} \stackrel{!}{=} 0 \qquad
\frac{\partial^2 \phi(\vec{x}, \vec{x_0})}{\partial z_0^2}\bigg|_{\vec{x}_0=\vec{x}_0^*} \neq 0.
\end{align}
%
\subsubsection{First Derivative of Phase w.r.t. Height}
%
\begin{align}
\label{eq:PhaseSGDer1Condition}
\frac{\partial \phi(\vec{x}, \vec{x_0})}{\partial z_0} = \frac{\partial \phi_{S}(\vec{x})}{\partial z_0}   + \frac{\partial \phi_{G}(\vec{x}-\vec{x}_0)}{\partial z_0} \stackrel{!}{=} 0
\end{align}
%
The phase term and its derivative for the primary point source located at $\vec{x}_S$ radiating the primary sound field is given as
\begin{align}
\label{eq:phaseS}
\phi_{S}(\vec{x}) = -\wc |\vec{x}_0-\vec{x}_S|\qquad \frac{\partial \phi_{S}(\vec{x})}{\partial z_0} = - \wc \frac{z_0-z_S}{|\vec{x}_0-\vec{x}_S|} = + \wc \frac{z_S-z_0}{|\vec{x}_S-\vec{x}_0|}
\end{align}
at the SSD.
The phase term and its derivative for the Green's function at $\vec{x_0}$ w.r.t. to listening point $\vec{x}$ is given as
\begin{align}
\label{eq:phaseG}
\phi_{G}(\vec{x}) = -\wc |\vec{x}-\vec{x}_0| \qquad \frac{\partial \phi_{G}(\vec{x}-\vec{x}_0)}{\partial z_0} = +\wc \frac{z-z_0}{|\vec{x}-\vec{x}_0|} = -\wc \frac{z_0-z}{|\vec{x}_0-\vec{x}|}
\end{align}
In the general case, the Cartesian local wave vectors are derived by evaluating the complete gradient $\vec{k} = \stackrel{!}{-} \nabla_{\vec{x}_0} \phi$ from above phases, i.e.
\begin{align}
\label{eq:WaveVectorsSPA1}
\vec{k}_S = + \wc \frac{\vec{x}_0-\vec{x}_S}{|\vec{x}_0-\vec{x}_S|}
\qquad \vec{k}_G^I = + \wc \frac{\vec{x}_0-\vec{x}}{|\vec{x}_0-\vec{x}|}
\qquad \vec{k}_G \stackrel{!}{=} - \vec{k}_G^I = - \wc \frac{\vec{x}_0-\vec{x}}{|\vec{x}_0-\vec{x}|} = + \wc \frac{\vec{x}-\vec{x}_0}{|\vec{x}-\vec{x}_0|}.
\end{align}
We see that $\vec{k}_S $ is a vector with direction from $\vec{x}_S$ to $\vec{x}_0$. On the other hand, $\vec{k}_G^I$ is a vector with direction from $\vec{x}$ to $\vec{x}_0$. The introduced (help) vector 
$\vec{k}_G$ is a vector from $\vec{x}_0$ to $\vec{x}$ (opposed to $\vec{k}_G^I $ as defined) and will be useful later.

Similarly to \eqref{eq:PhaseSGDer1Condition}, the SPA demands the local wave vectors condition 
\begin{align}
-\vec{k}_S - \vec{k}_G^I \stackrel{!}{=} 0
\end{align}
to be fulfilled in the stationary point.
This is achieved when $\vec{k}_S $ and $\vec{k}_G^I$ are opposed to each other, respectively the coincidence
\begin{align}
\vec{k}_S \stackrel{!}{=} - \vec{k}_G^I = + \vec{k}_G
\end{align}
holds.
Here $\vec{k}_G$ becomes helpful:
From above equation we can state that $\vec{k}_S $ and $\vec{k}_G$ are coincident in the stationary point. Firtha uses the notation $\vec{k}_G = \vec{k}_G(\vec{x}-\vec{x}_0^*(\vec{x}))$ and $\vec{k}^I_G=\vec{k}_G(\vec{x}_0^*(\vec{x})-\vec{x})$ with reciprocity mindset (i.e. take $\partial / \partial z$ for Green's function to obtain $\vec{k}_G$ directly, see \eqref{eq:kSrkGr}) and also states \cite[(7)]{Firtha2018}, \cite[(3.34)]{Firtha2018Diss}, \cite[Fig. 3.7]{Firtha2018Diss}
\begin{align}
\label{eq:SPA_condition_z}
\vec{k}_S(\vec{x}_0^*(\vec{x})) = \vec{k}_G(\vec{x}-\vec{x}_0^*(\vec{x})) = -\vec{k}_G(\vec{x}_0^*(\vec{x})-\vec{x}).
\end{align}


Before proceeding, we introduce the shortened notation 
%
\begin{align}
&\phi'_{S,z}(\vec{x}_0^*(\vec{x})) = \frac{\partial \phi_{S}(\vec{x})}{\partial z_0}\bigg|_{\vec{x}_0=\vec{x}_0^*}\qquad
&\phi'_{G,z}(\vec{x}-\vec{x}_0^*(\vec{x})) = \frac{\partial \phi_{G}(\vec{x}-\vec{x}_0)}{\partial z_0}\bigg|_{\vec{x}_0=\vec{x}_0^*} \\
&\phi''_{S,zz}(\vec{x}_0^*(\vec{x})) = \frac{\partial^2 \phi_{S}(\vec{x})}{\partial z_0^2}\bigg|_{\vec{x}_0=\vec{x}_0^*}\qquad
&\phi''_{G,zz}(\vec{x}-\vec{x}_0^*(\vec{x})) = \frac{\partial^2 \phi_{G}(\vec{x}-\vec{x}_0)}{\partial z_0^2}\bigg|_{\vec{x}_0=\vec{x}_0^*}
\end{align}
for the first and second derivatives of the phases, since these are required for the SPA.
Note that the shortened version omits the index $_0$, so keep \eqref{eq:PntSrcDerivEqualities} in mind.

\subsubsection{Second Derivative of Phase w.r.t. z}
For the primary point source example we get
\begin{align}
&\frac{\partial \phi_{S}(\vec{x})}{\partial z_0} = - \wc \frac{z_0-z_S}{|\vec{x}_0-\vec{x}_S|}\qquad
\frac{\partial^2 \phi_{S}(\vec{x})}{\partial z_0^2} = \frac{\partial}{\partial z_0}[- \wc \frac{z_0-z_S}{|\vec{x}_0-\vec{x}_S|}]\\
&\frac{\partial^2 \phi_{S}(\vec{x})}{\partial z_0^2} = \frac{-\wc |\vec{x}_0-\vec{x}_S|^2 + \wc (z_0-z_S)^2}{|\vec{x}_0-\vec{x}_S|^3}\\
&\frac{\partial \phi_{G}(\vec{x}-\vec{x}_0)}{\partial z_0} = +\wc \frac{z-z_0}{|\vec{x}-\vec{x}_0|}\qquad
\frac{\partial^2 \phi_{G}(\vec{x}-\vec{x}_0)}{\partial z_0^2} = \frac{\partial}{\partial z_0}[\wc \frac{z-z_0}{|\vec{x}-\vec{x}_0|}]\\
&\frac{\partial^2 \phi_{G}(\vec{x}-\vec{x}_0)}{\partial z_0^2} = \frac{-\wc |\vec{x}-\vec{x}_0|^2 + \wc (z-z_0)^2}{|\vec{x}-\vec{x}_0|^3}
\end{align}
The stationary point (actually the height) $z_0^* = 0$ holds when considering the listening plane in the $xy$-plane ($z=0$) and sound fields that exhibit local wave vectors $k_{S,z}(\vec{x}_0) = 0$ (such as point sources located within the $xy$-plane or plane waves traveling only with $k_x$ and $k_y$ components).
These two assumptions trivially fulfill \eqref{eq:PhaseSGDer1Condition}, yielding $\vec{x}_0^*=(x_0^*,y_0^*,0)^\mathrm{T}$ by considering $\vec{x}=(x,y,0)^\mathrm{T}$, $\vec{x}_S=(x_S,y_S,0)^\mathrm{T}$.
We can also recap this in \eqref{eq:WaveVectorsSPA1} for $z_S=0$ and $z=0$ requiring $z_0=z_0^*=0$.
In this stationary point we can simplify (cf. \cite[eq. (10)ff]{Firtha2018})
\begin{align}
\label{eq:phizz}
\frac{\partial^2 \phi_{S}(\vec{x}_0-\vec{x}_S)}{\partial z_0^2}\bigg|_{z_0=z_0^*} = \frac{-\wc}{|\vec{x}_0^*-\vec{x}_S|}\qquad
\frac{\partial^2 \phi_{G}(\vec{x}-\vec{x}_0)}{\partial z_0^2}\bigg|_{z_0=z_0^*} = \frac{-\wc}{|\vec{x}-\vec{x}_0^*|}
\end{align}


\subsubsection{Synthesis Integral from 1st SPA}
With the SPA \eqref{Eq:SPAintegral} $\rightarrow$ \eqref{Eq:SPAResult} we can reformulate \eqref{eq:HIE_FAR1} towards
%
\begin{align}
\label{eq:HIE_FAR1_SPA1}
S(\vec{x}) \approx \int\limits_{0}^{2\pi}
\im
& \sqrt{\frac{2\pi}{|\phi''_{S,zz}(\vec{x}_0^*(\vec{x}))+\phi''_{G,zz}(\vec{x}-\vec{x}_0^*(\vec{x}))|}} 
\left[
k_{S,r}(\vec{x}_0^*(\vec{x})) + k_{G,r}(\vec{x} - \vec{x}_0^*(\vec{x}))
\right]
S(\vec{x}_0^*(\vec{x}))\,G(\vec{x}-\vec{x}_0^*(\vec{x}))\cdot\nonumber\\
&\e^{\im \frac{\pi}{4}\,\mathrm{sgn}(\phi''_{S,zz}(\vec{x}_0^*(\vec{x}))+\phi''_{G,zz}(\vec{x}-\vec{x}_0^*(\vec{x})))}
R \, \fsd \azx_0
\end{align}
%
for the stationary point $z_0^*=0$ under the above assumptions.
%
The sgn() of phase term results in $-1$, cf. \eqref{eq:phizz}, \cite[(4.11)ff]{Firtha2018Diss} yielding
%
\begin{align}
\label{eq:HIE_FAR1_SPA2}
S(\vec{x}) \approx \int\limits_{0}^{2\pi}
\im
&\sqrt{\frac{2\pi}{|\phi''_{S,zz}(\vec{x}_0^*(\vec{x}))+\phi''_{G,zz}(\vec{x}-\vec{x}_0^*(\vec{x}))|}} 
\left[
k_{S,r}(\vec{x}_0^*(\vec{x})) + k_{G,r}(\vec{x} - \vec{x}_0^*(\vec{x}))
\right]\cdot\nonumber\\
&S(\vec{x}_0^*(\vec{x}))\,G(\vec{x}-\vec{x}_0^*(\vec{x}))
\e^{-\im \frac{\pi}{4}}
R \, \fsd \azx_0.
\end{align}
%
Rearranging $\e^{-\im \frac{\pi}{4}}$ and $\wc$ (typical in WFS literature)
%
\begin{align}
\label{eq:HIE_FAR1_SPA3}
S(\vec{x}) \approx \int\limits_{0}^{2\pi} &
\sqrt{\frac{2\pi}{\im |\phi''_{S,zz}(\vec{x}_0^*(\vec{x}))+\phi''_{G,zz}(\vec{x}-\vec{x}_0^*(\vec{x}))|}} 
\cdot \im \, \left[
k_{S,r}(\vec{x}_0^*(\vec{x})) + k_{G,r}(\vec{x} - \vec{x}_0^*(\vec{x}))
\right]\cdot\nonumber\\
&S(\vec{x}_0^*(\vec{x}))\,G(\vec{x}-\vec{x}_0^*(\vec{x}))
R \, \fsd \azx_0,
\end{align}
%
\begin{align}
\label{eq:HIE_FAR1_SPA4}
S(\vec{x}) \approx \int\limits_{0}^{2\pi}
\sqrt{\frac{2\pi}{\jwc}} & 
\sqrt{\frac{\wc}{|\phi''_{S,zz}(\vec{x}_0^*(\vec{x}))+\phi''_{G,zz}(\vec{x}-\vec{x}_0^*(\vec{x}))|}} 
\cdot \im \, \left[
k_{S,r}(\vec{x}_0^*(\vec{x})) + k_{G,r}(\vec{x} - \vec{x}_0^*(\vec{x}))
\right]\cdot\nonumber\\
&S(\vec{x}_0^*(\vec{x}))\,G(\vec{x}-\vec{x}_0^*(\vec{x}))
R \, \fsd \azx_0.
\end{align}
%
Rewritten with the vertical principal curvature $\kappa$ with $[\kappa] = 1/\text{m}$ and vertical principal radius $\rho=1/\kappa$ with $[\rho] = \text{m}$ of a wavefront (cf. \cite[(4.15)-(4.17), Fig. 3.2, Tab. 3.1]{Firtha2018Diss} we get
%
\begin{align}
\label{eq:HIE_FAR1_SPA5}
S(\vec{x}) \approx \int\limits_{0}^{2\pi} 
\sqrt{\frac{2\pi}{\jwc}} &
\sqrt{\frac{1}{\kappa_S^v(\vec{x}_0^*(\vec{x}))+\kappa_G^v(\vec{x}-\vec{x}_0^*(\vec{x}))}} 
\cdot \im \, \left[
k_{S,r}(\vec{x}_0^*(\vec{x})) + k_{G,r}(\vec{x} - \vec{x}_0^*(\vec{x}))
\right]\cdot\nonumber\\
&S(\vec{x}_0^*(\vec{x}))\,G(\vec{x}-\vec{x}_0^*(\vec{x}))
R \, \fsd \azx_0,
\end{align}
%
\begin{align}
\label{eq:HIE_FAR1_SPA6}
S(\vec{x}) \approx \int\limits_{0}^{2\pi}
\sqrt{\frac{2\pi}{\jwc}} &
\sqrt{\frac{\rho_S^v(\vec{x}_0^*(\vec{x}))\cdot\rho_G^v(\vec{x}-\vec{x}_0^*(\vec{x}))}{\rho_S^v(\vec{x}_0^*(\vec{x}))+\rho_G^v(\vec{x}-\vec{x}_0^*(\vec{x}))}} 
\cdot \im \, \left[
k_{S,r}(\vec{x}_0^*(\vec{x})) + k_{G,r}(\vec{x} - \vec{x}_0^*(\vec{x}))
\right]\cdot\nonumber\\
&S(\vec{x}_0^*(\vec{x}))\,G(\vec{x}-\vec{x}_0^*(\vec{x}))
R \, \fsd \azx_0.
\end{align}
%
From \eqref{eq:HIE_FAR1_SPA4}, \eqref{eq:HIE_FAR1_SPA5}, \eqref{eq:HIE_FAR1_SPA6}, cf. \cite[(4.15-4.17)]{Firtha2018Diss}
\begin{align}
&\sqrt{\frac{\wc}{|\phi''_{S,zz}(\vec{x}_0^*(\vec{x}))+\phi''_{G,zz}(\vec{x}-\vec{x}_0^*(\vec{x}))|}}=\\ 
&\sqrt{\frac{1}{\kappa_S^v(\vec{x}_0^*(\vec{x}))+\kappa_G^v(\vec{x}-\vec{x}_0^*(\vec{x}))}}=\\
&\sqrt{\frac{\rho_S^v(\vec{x}_0^*(\vec{x}))\cdot\rho_G^v(\vec{x}-\vec{x}_0^*(\vec{x}))}{\rho_S^v(\vec{x}_0^*(\vec{x}))+\rho_G^v(\vec{x}-\vec{x}_0^*(\vec{x}))}} 
\end{align}
with
\begin{align}
\phi_{S,zz}''(\vec{x}_0^*(\vec{x})) = \frac{-\wc}{|\vec{x}_0^*(\vec{x})-\vec{x}_S|}\qquad
\phi_{G,zz}''(\vec{x}-\vec{x}_0^*(\vec{x}))  = \frac{-\wc}{|\vec{x}-\vec{x}_0^*(\vec{x})|}
\end{align}
we state the vertical principal curvatures
\begin{align}
\kappa_S^v(\vec{x}_0^*(\vec{x})) = \frac{1}{|\vec{x}_0^*(\vec{x})-\vec{x}_S|}\qquad
\kappa_G^v(\vec{x}-\vec{x}_0^*(\vec{x})) = \frac{1}{|\vec{x}-\vec{x}_0^*(\vec{x})|}
\end{align}
and vertical principal radii
\begin{align}
\rho_S^v(\vec{x}_0^*(\vec{x})) = |\vec{x}_0^*(\vec{x})-\vec{x}_S|\qquad
\rho_G^v(\vec{x}-\vec{x}_0^*(\vec{x})) = |\vec{x}-\vec{x}_0^*(\vec{x})|
\end{align}
for the initiated primary point source example.
%

\subsubsection{Synthesis Integral Towards 2.5D Unified WFS}
Due to \eqref{eq:SPA_condition_z}, i.e. coincidence of local wavevectors around the stationary point (we can simplify with factor 2) and accounting for 'visible' elements only (we can introduce a spatial window function, cf. \textit{Application example \#2: The Kirchhoff approximation} \cite[p.46]{Firtha2018Diss}) we modify \eqref{eq:HIE_FAR1_SPA4}
%
\begin{align}
\label{eq:HIE_FAR1_SPA7}
S(\vec{x}) \approx \int\limits_{0}^{2\pi} 
w(\vec{x}_0^*(\vec{x})) & \sqrt{\frac{8\pi}{\jwc}}
\sqrt{\frac{\wc}{|\phi''_{S,zz}(\vec{x}_0^*(\vec{x}))+\phi''_{G,zz}(\vec{x}-\vec{x}_0^*(\vec{x}))|}} \cdot \nonumber\\
& \,\im \, k_{S,r}(\vec{x}_0^*(\vec{x})) \,
S(\vec{x}_0^*(\vec{x}))\,G(\vec{x}-\vec{x}_0^*(\vec{x}))
R \, \fsd \azx_0
\end{align}
%
with
%
\begin{equation}\label{eq:as}
w(\vec{x}_0^*(\vec{x})) =
    \begin{cases}
      1 &  \text{if}\qquad
       \unitn(\vec{x}_0^*(\vec{x})) \cdot \vec{k}_S(\vec{x}_0^*(\vec{x})) \geq 0 \\
      0 & \text{otherwise}
    \end{cases}.
\end{equation}
%
Now we could define a relation between $\vec{x}_0^*=(r_0^*,\azx_0^*,0)^\mathrm{T}$ and $\vec{x}=(r,\azx,0)^\mathrm{T}$ to extract the WFS driving function $D_\text{2.5D WFS}(\vec{x}_0)$.
Typically, the functional dependence $\vec{x}_0^*(\vec{x})$ is exchanged to $\vec{x}_\text{Ref}(\vec{x}_0)$, defining a reference curve, such that for each $\vec{x}_0$ exists a certain position $\vec{x}_\text{Ref}$ at which amplitude-correct SFS is ensured.
This was introduced as \textbf{unified WFS framework} in \cite[Sec. IIIB]{Firtha2017} using the referencing function $d(\vec{x}_0)$ for defining positions of amplitude-correct synthesis (PCS).
The synthesis integral is then formulated as the single layer potential
%
\begin{align}
\label{eq:HIE_FAR1_SPA8}
S(\vec{x}) \approx & \int\limits_{0}^{2\pi}
\overbrace{\left[
w(\vec{x}_0)\,\sqrt{\frac{8\pi}{\jwc}}
\sqrt{\underbrace{\left[\frac{\wc}{|\phi''_{S,zz}(\vec{x}_0)+\phi''_{G,zz}(\vec{x}_\text{Ref}(\vec{x}_0)-\vec{x}_0)|}\right]}_{d(\vec{x}_0)}} 
\,\im \, k_{S,r}(\vec{x}_0) \,
S(\vec{x}_0)\right]}^{D_\text{2.5D WFS}(\vec{x}_0)}\,G(\vec{x}-\vec{x}_0)
R \, \fsd \azx_0,
\end{align}
%
which yields \textbf{unified 2.5D WFS} using local wave vector and local wavefront curvature concepts, cf. \cite[Sec. 4.1.3]{Firtha2018Diss}, here for the special case using a circular SSD in the synthesis integral.

 \subsection{Stationary Phase Approximation w.r.t. Azimuth}
\fscom{NOTE: The calculus in this subsection, although resulting in something meaningful, is most probably a wrong Ansatz for reducing Single Layer Potential with two SPAs in polar/cylindrical coordinates. We have to check for consistency with the Hessian matrix approach from \cite[App. B]{Firtha2018Diss} to make sure.}

In order to get an approximation of the sound field at a position $S(\vec{x})$ (we will see that this position is dependent of secondary source and primary source positions) we can introduce a second SPA to further reduce the integrals dimensionality.
%
For the SPA w.r.t. $\azx_0$ the phase term of \eqref{eq:HIE_FAR1_SPA4} must now fulfill
%
\begin{align}
\frac{\partial \phi(\vec{x}, \vec{x_0})}{\partial \azx_0}\bigg|_{\vec{x}_0=\vec{x}_0^*} \stackrel{!}{=} 0\qquad 
\frac{\partial^2 \phi(\vec{x}, \vec{x_0})}{\partial \azx_0^2}\bigg|_{\vec{x}_0=\vec{x}_0^*} \neq 0
\end{align}
for the stationary angle $\azx_0^*$.
%
Again, short notation (omitting $_0$ index again)
%
\begin{align}
\phi'_{S,\azx}(\vec{x}_0^*(\vec{x})) = \frac{\partial \phi_{S}(\vec{x})}{\partial \azx_0}\bigg|_{\vec{x}_0=\vec{x}_0^*}\qquad
\phi'_{G,\azx}(\vec{x}-\vec{x}_0^*(\vec{x})) = \frac{\partial \phi_{G}(\vec{x}-\vec{x}_0)}{\partial \azx_0}\bigg|_{\vec{x}_0=\vec{x}_0^*} \\
\phi''_{S,\azx\azx}(\vec{x}_0^*(\vec{x})) = \frac{\partial^2 \phi_{S}(\vec{x})}{\partial \azx_0^2}\bigg|_{\vec{x}_0=\vec{x}_0^*}\qquad
\phi''_{G,\azx\azx}(\vec{x}-\vec{x}_0^*(\vec{x})) = \frac{\partial^2 \phi_{G}(\vec{x}-\vec{x}_0)}{\partial \azx_0^2}\bigg|_{\vec{x}_0=\vec{x}_0^*}
\end{align}
%
will be used later.
%
\subsubsection{First Derivative of Phase w.r.t. Azimuth}
We derive $\frac{\partial \phi(\vec{x}, \vec{x_0})}{\partial \azx_0}\big|_{\vec{x}_0=\vec{x}_0^*} \stackrel{!}{=} 0$ with the above initiated example of a target point source.
For that we consider the remaining 2D polar coordinates (for each $\vec{x}$-related vector the $z$-component is zero as demanded by the first SPA above), thus
\begin{align}
x_0 = r_0 \cos\azx_0\qquad x = r \cos\azx\qquad y_0 = r_0 \sin\azx_0\qquad y = r \sin\azx.
\end{align}
The phase terms
$\phi_G = -\wc |\vec{x} -\vec{x}_0|$ and $\phi_S = -\wc |\vec{x}_0 - \vec{x}_S|$ (cf. \eqref{eq:phaseG}, \eqref{eq:phaseS}) recast in polar coordinates, yields
\begin{align}
\phi_G =& -\wc \sqrt{(r \cos\azx-r_0 \cos\azx_0)^2 + (r \sin\azx-r_0 \sin\azx_0)^2}\\
\phi_S =& -\wc \sqrt{(r_0 \cos\azx_0-r_S \cos\azx_S)^2 + (r_0 \sin\azx_0-r_S \sin\azx_S)^2}
\end{align}
and expanding for convenient derivation w.r.t. $\azx_0$:
\begin{align}
\phi_G =& -\wc \sqrt{r^2 \cos^2\azx + r_0^2 \cos^2\azx_0 -2 r r_0 \cos\azx \cos\azx_0 + r^2 \sin^2\azx + r_0^2 \sin^2\azx_0 -2 r r_0 \sin\azx \sin\azx_0}\\
=& -\wc \sqrt{r^2 + r_0^2 - 2 r r_0 (\cos\azx \cos\azx_0 + \sin\azx \sin\azx_0)}\\
\phi_S =& -\wc \sqrt{r_0^2 + r_S^2 - 2 r_0 r_S (\cos\azx_0 \cos\azx_S + \sin\azx_0 \sin\azx_S)}
\end{align}
\begin{align}
\frac{\partial \phi_G}{\partial \azx_0} =& -\wc \frac{- 2 r r_0 (-\cos\azx\sin\azx_0+\sin\azx \cos\azx_0)}{2\cdot \sqrt{r^2 + r_0^2 - 2 r r_0 (\cos\azx \cos\azx_0 + \sin\azx \sin\azx_0)}}\\
=& + \wc \frac{r r_0 (-\cos\azx\sin\azx_0+\sin\azx \cos\azx_0)}{|\vec{x} -\vec{x}_0|}\\
=& + \wc \frac{r r_0 \sin(\azx-\azx_0)}{|\vec{x} -\vec{x}_0|}\\
\frac{\partial \phi_S}{\partial \azx_0} =& -\wc \frac{- 2 r_0 r_S (-\sin\azx_0\cos\azx_S+\cos\azx_0\sin\azx_S)}{2\cdot \sqrt{r_0^2 + r_S^2 - 2 r_0 r_S (\cos\azx_0 \cos\azx_S + \sin\azx_0 \sin\azx_S)}}\\
=& +\wc \frac{r_0 r_S (-\sin\azx_0\cos\azx_S+\cos\azx_0\sin\azx_S)}{|\vec{x}_0 - \vec{x}_S|}\\
=& -\wc \frac{r_0 r_S \sin(\azx_0 - \azx_S)}{|\vec{x}_0 - \vec{x}_S|}
\end{align}
Thus for $\frac{\partial \phi_G}{\partial \azx_0}\big|_{\azx_0=\azx_0^*} + \frac{\partial \phi_S}{\partial \azx_0}\big|_{\azx_0=\azx_0^*} \stackrel{!}{=} 0$ we must demand
\begin{align}
\underbrace{+\wc \frac{r r_0 \sin(\azx-\azx_0)}{|\vec{x} -\vec{x}_0|}}_{\propto -k_{G,\azx}^I} + \underbrace{-\wc \frac{r_0 r_S \sin(\azx_0 - \azx_S)}{|\vec{x}_0 - \vec{x}_S|}}_{\propto -k_{S,\azx}} \stackrel{!}{=} 0\qquad\rightarrow\qquad
\underbrace{\frac{r \sin(\azx-\azx_0)}{|\vec{x} -\vec{x}_0|}}_{\propto -k_{G,\azx}^I = +k_{G,\azx}} \stackrel{!}{=} \underbrace{\frac{r_S \sin(\azx_0 - \azx_S)}{|\vec{x}_0 - \vec{x}_S|}}_{\propto +k_{S,\azx}}
\end{align}
using the above introduced local wave vector definitions.
Only (pretty sure about that, but not strictly proven) the $2\pi$-periodic sine terms (stationary point location is $2\pi$-periodic as expected) can result in equality of the latter equation.
The condition is fulfilled (simply yielding $0 \stackrel{!}{=} 0$), when $\azx_0 = \azx_S$ together with either $\azx = \azx_0$ or $\azx = \azx_0 + \pi$, which proves the trivial fact that if the source, receiver and SSD positions are located along a straight line, the intersection at the SSD (if any, the shorter one to xPS, \fscom{check, this might indicate windowing}) is the stationary point. 
Thus, formally 
\begin{align}
\label{eq:SPA_condition_azx}
k_{S,\azx}(\azx_0^*(\azx)) = k_{G,\azx}(\azx - \azx_0^*(\azx))
\end{align}
holds.

\subsubsection{Second Derivative of Phase}
From above we recall the first derivatives and prepare the second derivatives
\begin{align}
\frac{\partial \phi_G}{\partial \azx_0} =+\wc \frac{r r_0 \sin(\azx-\azx_0)}{|\vec{x} -\vec{x}_0|}\qquad \rightarrow \qquad
&\frac{\partial^2 \phi_G}{\partial \azx_0^2} =+\frac{\partial }{\partial \azx_0} [ \wc \frac{r r_0 \sin(\azx-\azx_0)}{|\vec{x} -\vec{x}_0|} ] \\
\frac{\partial \phi_S}{\partial \azx_0} =-\wc \frac{r_0 r_S \sin(\azx_0 - \azx_S)}{|\vec{x}_0 - \vec{x}_S|} \qquad \rightarrow \qquad
&\frac{\partial^2 \phi_S}{\partial \azx_0^2} =-\frac{\partial}{\partial \azx_0} [ \wc \frac{r_0 r_S \sin(\azx_0 - \azx_S)}{|\vec{x}_0 - \vec{x}_S|} ].
\end{align}
With $(\frac{u}{v})' = \frac{u'v-uv'}{v^2}$ and sin/cos addition theorems we have for $\phi_G$:
\begin{align}
u =&  + r r_0 \sin(\azx-\azx_0)\\
u' =& - r r_0 \cos(\azx-\azx_0)\\
v =& |\vec{x} -\vec{x}_0| = \sqrt{r^2 + r_0^2 - 2 r r_0 (\cos\azx \cos\azx_0 + \sin\azx \sin\azx_0)}\\
v =& |\vec{x} -\vec{x}_0| = \sqrt{r^2 + r_0^2 - 2 r r_0 \cos(\azx - \azx_0)}\\
v' =& \frac{- 2 r r_0 (-\cos\azx \sin\azx_0 + \sin\azx \cos\azx_0)}{2\sqrt{r^2 + r_0^2 - 2 r r_0 (\cos\azx \cos\azx_0 + \sin\azx \sin\azx_0)}}\\
v' =& \frac{- r r_0 \sin(\azx-\azx_0)}{\sqrt{r^2 + r_0^2 - 2 r r_0 \cos(\azx-\azx_0)}} = \frac{- r r_0 \sin(\azx-\azx_0)}{|\vec{x} -\vec{x}_0|}
\end{align}
Bringing together $(\frac{u}{v})'$ with $\wc$:
\begin{align}
\frac{\partial^2 \phi_G}{\partial^2 \azx_0} =&+\wc \frac{
-r r_0 \cos(\azx-\azx_0) |\vec{x} -\vec{x}_0| -
(r r_0 \sin(\azx-\azx_0) \frac{- r r_0 \sin(\azx-\azx_0)}{|\vec{x} -\vec{x}_0|})
}
{|\vec{x} -\vec{x}_0|^2} \\
\frac{\partial^2 \phi_G}{\partial^2 \azx_0} =&+\wc \frac{
- r r_0 \cos(\azx-\azx_0) |\vec{x} -\vec{x}_0|^2-
(r r_0 \sin(\azx-\azx_0) (- r r_0 \sin(\azx-\azx_0))
)}
{|\vec{x} -\vec{x}_0|^3}\\
\frac{\partial^2 \phi_G}{\partial^2 \azx_0} =&+\wc \frac{
- r r_0 \cos(\azx-\azx_0) |\vec{x} -\vec{x}_0|^2 + 
r^2 r_0^2 \sin^2(\azx-\azx_0)
}
{|\vec{x} -\vec{x}_0|^3}\\
\frac{\partial^2 \phi_G}{\partial^2 \azx_0} =&+\wc \frac{
- r r_0 \cos(\azx-\azx_0) |\vec{x} -\vec{x}_0|^2 + 
r^2 r_0^2 (1-\cos^2(\azx-\azx_0))
}
{|\vec{x} -\vec{x}_0|^3}
\end{align}
%
With $|\vec{x} -\vec{x}_0|^2 = {r^2 + r_0^2 -2 r r_0 \cos(\azx-\azx_0)}$
%
\begin{align}
\frac{\partial^2 \phi_G}{\partial^2 \azx_0} =&+\wc \frac{
- r r_0 \cos(\azx-\azx_0) (r^2 + r_0^2 -2 r r_0 \cos(\azx-\azx_0)) + 
r^2 r_0^2 (1-\cos^2(\azx-\azx_0))
}
{|\vec{x} -\vec{x}_0|^3}\\
\frac{\partial^2 \phi_G}{\partial^2 \azx_0} =&+\wc r r_0 \frac{
- \cos(\azx-\azx_0) (r^2 + r_0^2 -2 r r_0 \cos(\azx-\azx_0)) + 
r r_0 (1-\cos^2(\azx-\azx_0))
}
{|\vec{x} -\vec{x}_0|^3}\\
\frac{\partial^2 \phi_G}{\partial^2 \azx_0} =&+\wc r r_0 \frac{
-r^2 \cos(\azx-\azx_0) - r_0^2 \cos(\azx-\azx_0) + 2 r r_0 \cos^2(\azx-\azx_0) + 
r r_0 - r r_0 \cos^2(\azx-\azx_0)
}
{|\vec{x} -\vec{x}_0|^3}\\
\frac{\partial^2 \phi_G}{\partial^2 \azx_0} =&+\wc r r_0 \frac{
-r^2 \cos(\azx-\azx_0) - r_0^2 \cos(\azx-\azx_0) + r r_0 \cos^2(\azx-\azx_0) + 
r r_0
}
{|\vec{x} -\vec{x}_0|^3}
\end{align}
This yields
\begin{align}
\frac{\partial^2 \phi_G}{\partial^2 \azx_0} =&+\wc r r_0 \frac{
(r \cos(\azx-\azx_0) - r_0) \cdot (r_0 \cos(\azx-\azx_0) - r)
}
{|\vec{x} -\vec{x}_0|^3}.
\end{align}
For $\phi_S$ a similar calculus yields
\begin{align}
\frac{\partial^2 \phi_S}{\partial \azx_0^2} =&+\wc r_0 r_S \frac{
(r_0 \cos(\azx_0-\azx_S) - r_S) \cdot (r_S \cos(\azx_0-\azx_S) - r_0)
}
{|\vec{x}_0 -\vec{x}_S|^3}.
\end{align}
In the stationary point all cosine terms become unity, i.e.
\begin{align}
\frac{\partial^2 \phi_G}{\partial^2 \azx_0} =&+\wc r r_0 \frac{
(r - r_0) \cdot (r_0 - r)
}
{|\vec{x} -\vec{x}_0|^3} = -\wc r r_0 \frac{
(r - r_0)^2 
}
{|\vec{x} -\vec{x}_0|^3}=
-\wc \frac{r r_0}{|\vec{x} -\vec{x}_0|}\\
%
\frac{\partial^2 \phi_S}{\partial \azx_0^2} =&+\wc r_0 r_S \frac{
(r_0 - r_S) \cdot (r_S - r_0)
}
{|\vec{x}_0 -\vec{x}_S|^3} = -\wc r_0 r_S \frac{
(r_0 - r_S)^2
}
{|\vec{x}_0 -\vec{x}_S|^3} = -\wc \frac{
r_0 r_S
}
{|\vec{x}_0 -\vec{x}_S|}
\end{align}



\subsubsection{Synthesis Integral from 2nd SPA}
\begin{align}
S(\vec{x}) \approx \sqrt{\frac{2\pi}{\jwc}}  
&
\sqrt{\frac{\wc}{|\phi''_{S,zz}(\vec{x}_0^*(\vec{x}))+\phi''_{G,zz}(\vec{x}-\vec{x}_0^*(\vec{x}))|}} 
\cdot \im \, \left[
k_{S,r}(\vec{x}_0^*(\vec{x})) + k_{G,r}(\vec{x} - \vec{x}_0^*(\vec{x}))
\right]\cdot\nonumber\\
&
\sqrt{\frac{2\pi}{|\phi''_{S,\azx\azx}(\vec{x}_0^*(\vec{x}))+\phi''_{G,\azx\azx}(\vec{x}-\vec{x}_0^*(\vec{x}))|}}
\e^{\im \frac{\pi}{4}\,\mathrm{sgn}(\phi''_{S,\azx\azx}(\vec{x}_0^*(\vec{x}))+\phi''_{G,\azx\azx}(\vec{x}-\vec{x}_0^*(\vec{x})))}\nonumber\\
&
S(\vec{x}_0^*(\vec{x}))\,G(\vec{x}-\vec{x}_0^*(\vec{x})) R
\end{align}
For sgn = -1
\begin{align}
S(\vec{x}) \approx 
&
\sqrt{\frac{2\pi}{\jwc}}  
\sqrt{\frac{\wc}{|\phi''_{S,zz}(\vec{x}_0^*(\vec{x}))+\phi''_{G,zz}(\vec{x}-\vec{x}_0^*(\vec{x}))|}} 
\cdot \im \, \left[
k_{S,r}(\vec{x}_0^*(\vec{x})) + k_{G,r}(\vec{x} - \vec{x}_0^*(\vec{x}))
\right]\cdot\nonumber\\
&
\sqrt{\frac{2\pi}{\jwc}}
\sqrt{\frac{\wc}{|\phi''_{S,\azx\azx}(\vec{x}_0^*(\vec{x}))+\phi''_{G,\azx\azx}(\vec{x}-\vec{x}_0^*(\vec{x}))|}}\nonumber\\
&
S(\vec{x}_0^*(\vec{x}))\,G(\vec{x}-\vec{x}_0^*(\vec{x})) R
\end{align}
$\sqrt{\frac{2\pi}{\jwc}}$ handling:
\begin{align}
S(\vec{x}) \approx \frac{2\pi}{\jwc}R \cdot
&  
\sqrt{\frac{\wc}{|\phi''_{S,zz}(\vec{x}_0^*(\vec{x}))+\phi''_{G,zz}(\vec{x}-\vec{x}_0^*(\vec{x}))|}} 
\cdot \im \, \left[
k_{S,r}(\vec{x}_0^*(\vec{x})) + k_{G,r}(\vec{x} - \vec{x}_0^*(\vec{x}))
\right]\cdot\nonumber\\
&
\sqrt{\frac{\wc}{|\phi''_{S,\azx\azx}(\vec{x}_0^*(\vec{x}))+\phi''_{G,\azx\azx}(\vec{x}-\vec{x}_0^*(\vec{x}))|}}\nonumber\\
&
S(\vec{x}_0^*(\vec{x}))\,G(\vec{x}-\vec{x}_0^*(\vec{x}))
\end{align}
canceling $\wc$ 
\begin{align}
S(\vec{x}) \approx \frac{2\pi}{\im}R \cdot
&  
\sqrt{\frac{1}{|\phi''_{S,zz}(\vec{x}_0^*(\vec{x}))+\phi''_{G,zz}(\vec{x}-\vec{x}_0^*(\vec{x}))|}} 
\cdot \im \, \left[
k_{S,r}(\vec{x}_0^*(\vec{x})) + k_{G,r}(\vec{x} - \vec{x}_0^*(\vec{x}))
\right]\cdot\nonumber\\
&
\sqrt{\frac{1}{|\phi''_{S,\azx\azx}(\vec{x}_0^*(\vec{x}))+\phi''_{G,\azx\azx}(\vec{x}-\vec{x}_0^*(\vec{x}))|}}\nonumber\\
&
S(\vec{x}_0^*(\vec{x}))\,G(\vec{x}-\vec{x}_0^*(\vec{x}))
\end{align}
Same handling as from \eqref{eq:HIE_FAR1_SPA6} to \eqref{eq:HIE_FAR1_SPA7}
\begin{align}
S(\vec{x}) \approx w(\vec{x}_0^*(\vec{x})) \frac{4\pi}{\im}R \cdot
&  
\sqrt{\frac{1}{|\phi''_{S,zz}(\vec{x}_0^*(\vec{x}))+\phi''_{G,zz}(\vec{x}-\vec{x}_0^*(\vec{x}))|}} 
\cdot \im \, 
k_{S,r}(\vec{x}_0^*(\vec{x}))
\cdot\nonumber\\
&
\sqrt{\frac{1}{|\phi''_{S,\azx\azx}(\vec{x}_0^*(\vec{x}))+\phi''_{G,\azx\azx}(\vec{x}-\vec{x}_0^*(\vec{x}))|}}\nonumber\\
&
S(\vec{x}_0^*(\vec{x}))\,G(\vec{x}-\vec{x}_0^*(\vec{x}))
\end{align}
which has then a similar look as \cite[(38)]{Firtha2018} (this solves in Cartesian coordinates!), here we have additional factor $R$\fscom{?!?!}.
Unit check of this equation with all above phase derivatives is consistent, though.
We have an SPA based propagation law from sound field $S(\vec{x}_0^*(\vec{x}))$ towards $S(\vec{x})$.


We insert the SPA I and II
\begin{align}
S(\vec{x}) \approx 4 \pi r_0 \cdot
&  
\sqrt{\frac{1}{|\frac{-\wc}{|\vec{x}_0-\vec{x}_S|}+\frac{-\wc}{|\vec{x}-\vec{x}_0|}|}} 
\cdot \, 
k_{S,r}(\vec{x}_0^*(\vec{x}))
\cdot\nonumber\\
&
\sqrt{\frac{1}{|-\wc \frac{r_0 r_S}{|\vec{x}_0 -\vec{x}_S|}-\wc \frac{r r_0}{|\vec{x} -\vec{x}_0|}|}}\nonumber\\
&
S(\vec{x}_0^*(\vec{x}))\,G(\vec{x}-\vec{x}_0^*(\vec{x}))
\end{align}
and insert for $k_{S,r}$ in the stationary point
\begin{align}
k_{S,r}(\vec{x_0}) = \wc
\end{align}
\begin{align}
S(\vec{x}) \approx 4 \pi r_0 \cdot
&  
\sqrt{\frac{1}{|\frac{-\wc}{|\vec{x}_0-\vec{x}_S|}+\frac{-\wc}{|\vec{x}-\vec{x}_0|}|}} 
\cdot \, 
\wc
\cdot\nonumber\\
&
\sqrt{\frac{1}{|-\wc \frac{r_0 r_S}{|\vec{x}_0 -\vec{x}_S|}-\wc \frac{r r_0}{|\vec{x} -\vec{x}_0|}|}}\nonumber\\
&
S(\vec{x}_0^*(\vec{x}))\,G(\vec{x}-\vec{x}_0^*(\vec{x}))
\end{align}
and for $S(\vec{x}_0^*(\vec{x}))\,G(\vec{x}-\vec{x}_0^*(\vec{x}))$  in the stationary point
\begin{align}
&S(\vec{x}_0^*(\vec{x})) = \frac{1}{4\pi} \frac{\e^{-\jwc |\vec{x}_0-\vec{x}_s|}}{|\vec{x}_0-\vec{x}_s|}\\
&G(\vec{x}-\vec{x}_0^*(\vec{x})) = \frac{1}{4\pi} \frac{\e^{-\jwc |\vec{x}-\vec{x}_0|}}{|\vec{x}-\vec{x}_0|}
\end{align}

\begin{align}
S(\vec{x}) \approx 4 \pi r_0 \cdot
&  
\sqrt{\frac{1}{|\frac{-\wc}{|\vec{x}_0-\vec{x}_S|}+\frac{-\wc}{|\vec{x}-\vec{x}_0|}|}} 
\cdot \, 
\wc
\cdot\nonumber\\
&
\sqrt{\frac{1}{|-\wc \frac{r_0 r_S}{|\vec{x}_0 -\vec{x}_S|}-\wc \frac{r r_0}{|\vec{x} -\vec{x}_0|}|}}\nonumber\\
&\frac{1}{4\pi} \frac{\e^{-\jwc |\vec{x}_0-\vec{x}_s|}}{|\vec{x}_0-\vec{x}_s|}\nonumber\\
&\frac{1}{4\pi} \frac{\e^{-\jwc |\vec{x}-\vec{x}_0|}}{|\vec{x}-\vec{x}_0|}
\end{align}
Remove $4\pi$ and $\wc$
\begin{align}
S(\vec{x}) \approx r_0 \cdot
&  
\sqrt{\frac{1}{|\frac{1}{|\vec{x}_0-\vec{x}_S|}+\frac{1}{|\vec{x}-\vec{x}_0|}|}} 
\sqrt{\frac{1}{|\frac{r_0 r_S}{|\vec{x}_0 -\vec{x}_S|}+\frac{r r_0}{|\vec{x} -\vec{x}_0|}|}}\nonumber\\
&\frac{\e^{-\jwc |\vec{x}_0-\vec{x}_s|}}{|\vec{x}_0-\vec{x}_s|}
\frac{1}{4\pi} \frac{\e^{-\jwc |\vec{x}-\vec{x}_0|}}{|\vec{x}-\vec{x}_0|}
\end{align}
We realize that $|\vec{x}_0-\vec{x}_s| + |\vec{x} -\vec{x}_0| = |\vec{x} -\vec{x}_s|$ in stationary point. This yields
\begin{align}
S(\vec{x}) \approx r_0 \cdot
&  
\sqrt{\frac{1}{|\frac{1}{|\vec{x}_0-\vec{x}_S|}+\frac{1}{|\vec{x}-\vec{x}_0|}|}} 
\sqrt{\frac{1}{|\frac{r_0 r_S}{|\vec{x}_0 -\vec{x}_S|}+\frac{r r_0}{|\vec{x} -\vec{x}_0|}|}}\nonumber\\
&\frac{\e^{-\jwc |\vec{x}-\vec{x}_s|}}{|\vec{x}_0-\vec{x}_s|\cdot |\vec{x}-\vec{x}_0|}
\frac{1}{4\pi} 
\end{align}
Rearranging fractions
\begin{align}
S(\vec{x}) \approx r_0 \cdot
&  
\sqrt{\frac{|\vec{x}_0-\vec{x}_S|\cdot |\vec{x}-\vec{x}_0|}{|\vec{x}_0-\vec{x}_S| + |\vec{x}-\vec{x}_0|}} \cdot
\sqrt{\frac{|\vec{x}_0-\vec{x}_S|\cdot |\vec{x}-\vec{x}_0|}{r_0 r_S |\vec{x} -\vec{x}_0| + r_0 r |\vec{x}_0-\vec{x}_S|}}
\nonumber\\
&\frac{\e^{-\jwc |\vec{x}-\vec{x}_s|}}{|\vec{x}_0-\vec{x}_s|\cdot |\vec{x}-\vec{x}_0|}
\frac{1}{4\pi} 
\end{align}
cancels out
\begin{align}
S(\vec{x}) \approx r_0 \cdot
&  
\sqrt{\frac{1}{|\vec{x}_0-\vec{x}_S| + |\vec{x}-\vec{x}_0|}} \cdot
\sqrt{\frac{1}{r_0 r_S |\vec{x} -\vec{x}_0| + r_0 r |\vec{x}_0-\vec{x}_S|}}
\nonumber\\
&\e^{-\jwc |\vec{x}-\vec{x}_s|}
\frac{1}{4\pi} 
\end{align}
\begin{align}
S(\vec{x}) \approx r_0 \cdot
&  
\sqrt{\frac{1}{|\vec{x}_0-\vec{x}_S| + |\vec{x}-\vec{x}_0|}} \cdot
\sqrt{\frac{|\vec{x}_0-\vec{x}_S| + |\vec{x}-\vec{x}_0|}{|\vec{x}_0-\vec{x}_S| + |\vec{x}-\vec{x}_0|}} 
\sqrt{\frac{1}{r_0 r_S |\vec{x} -\vec{x}_0| + r_0 r |\vec{x}_0-\vec{x}_S|}}
\nonumber\\
&\e^{-\jwc |\vec{x}-\vec{x}_s|}
\frac{1}{4\pi} 
\end{align}

\begin{align}
S(\vec{x}) \approx r_0 \cdot
\sqrt{\frac{|\vec{x}_0-\vec{x}_S| + |\vec{x}-\vec{x}_0|}{r_0 r_S |\vec{x} -\vec{x}_0| + r_0 r |\vec{x}_0-\vec{x}_S|}}
\frac{\e^{-\jwc |\vec{x}-\vec{x}_s|}}{4 \pi |\vec{x}-\vec{x}_s|} 
\end{align}

For $\vec{x} = (0,0,0)^\mathrm{T}$, $r = 0$, i.e. listening point in origin we obtain

\begin{align}
S(\vec{x}) \approx r_0 \cdot
\sqrt{\frac{|\vec{x}_0-\vec{x}_s| + |\vec{x}_0|}{r_0 r_S |\vec{x}_0|}}
\frac{\e^{-\jwc |\vec{x}-\vec{x}_s|}}{4 \pi |\vec{x}-\vec{x}_s|} 
\end{align}

\begin{align}
S(\vec{x}) \approx r_0 \cdot
\sqrt{\frac{r_s}{r^2_0 r_S}}
\frac{\e^{-\jwc |\vec{x}-\vec{x}_s|}}{4 \pi |\vec{x}-\vec{x}_s|} =
\frac{\e^{-\jwc |\vec{x}-\vec{x}_s|}}{4 \pi |\vec{x}-\vec{x}_s|}
\end{align}

as expected, i.e. phase and amplitude correct SFS of the target point source in this single location.


\section{Appendix A: Stationary Phase Approximation}
\fscom{Copy from \cite{Firtha2018} for convenience $\rightarrow$}
%
Both, the 2.5D WFS theory and the evaluation of Fourier integrals rely on the stationary phase approximation (SPA).
This method yields an approximate solution for integrals
\begin{equation}
\label{Eq:SPAintegral}
I = \int\limits_{-\infty}^{\infty} F(x) \, \e^{\im \phi(x)} \, \fsd x
\end{equation}
with real-valued $F,\phi$ that (i) contain at least one critical point in the integration path and (ii) considering highly oscillating $\e^{\im \phi(x)}$ compared to slowly varying $F(x)$. % \cite{Bleistein1984, Bleistein1986, Williams1999}.
The SPA is based on the Taylor series
\begin{align}
\phi(x) \approx \phi(x)\big|_{x=x^*} + \frac{(x-x^*)^2}{2} \frac{\partial^2 \phi(x)}{\partial x^2}\bigg|_{x=x^*} 
\end{align} 
of the phase function $\phi(x)$ around the critical point $x^*$, aka the stationary point, for which $\frac{\partial \phi(x)}{\partial x}\big|_{x=x^*} = 0$ and $\frac{\partial^2 \phi(x)}{\partial x^2}\big|_{x=x^*} \neq 0$ must hold for.
Under the given assumptions \eqref{Eq:SPAintegral} is approximated as %\cite[(2.7.18)]{Bleistein1984}
\begin{align}
\label{Eq:SPAResult}
I \approx \sqrt{\frac{2\pi}{| \frac{\partial^2 \phi(x)}{\partial x^2}\big|_{x=x^*}  |}} F(x^*) \, \e^{\im \phi(x^*) + \im \frac{\pi}{4}\,\mathrm{sgn}\left(  \frac{\partial^2 \phi(x)}{\partial x^2}\big|_{x=x^*}  \right)}.
\end{align}
For multidimensional integrals a generalized SPA formulation is available with the stationary point found, where the phase gradient vanishes. % \cite[(2.8.23)]{Bleistein1984}.
However, the integration can be evaluated by applying multiple one-dimensional SPAs consecutively under the present presumptions.
\fscom{$\leftarrow$ End of Copy from \cite{Firtha2018}}



\section{Appendix B: Circular Harmonics Expansion of Plane Wave 2.5D WFS Driving Function}


We check \cite[Ch. 4.4.2]{Ahrens2012} here in detail. Very helpful treatments on driving function expansions can be found in \cite{Hahn2016AES}.

The 2.5D unified WFS driving function for a plane wave referencing to single point $\mathbf{x}_{Ref}$ is known as, cf. \cite[2.177]{Schultz2016Diss}

\begin{align}
\label{eq:DPWWFS}
D_{PW,WFS}(\mathbf{x}_0,\omega) = w(\mathbf{x}_0) \sqrt{\frac{\mathrm{j \omega}}{c}} \sqrt{8 \pi} \cdot
\sqrt{|\mathbf{x}_{Ref}-\mathbf{x}_{0}|} \cdot
\langle \mathbf{\hat{k}}_{PW}, \mathbf{\hat{n}}(\mathbf{x}_0) \rangle \cdot
\mathrm{e}^{- \mathrm{j} \frac{\omega}{c}\langle \mathbf{\hat{k}}_{PW}, \mathbf{x}_0 \rangle}
\end{align}
using the $xy$-plane for SFS and referencing to origin as in \cite[Ch. 4.4.2]{Ahrens2012}, thus wave propagating direction
$\mathbf{\hat{k}}_{PW} = (\cos\phi_{PW}, \sin\phi_{PW},0)^\text{T}$, 
$\mathbf{x}_0=(r_0 \cos\phi_0, r_0 \sin\phi_0)^\text{T}$, 
$\mathbf{\hat{n}}(\mathbf{x}_0)=-(\cos\phi_0, \sin\phi_0)^\text{T}$,
$\mathbf{x}_{Ref} = (0,0,0)^\text{T} \rightarrow |\mathbf{x}_{Ref}-\mathbf{x}_{0}|=|\mathbf{x}_{0}|=r_0$.

We want to derive the Fourier series of the $\phi_0$-dependent terms in the above driving function. Once this is known, we can compare it with the Fourier series, which is inherently given in the 2.5D NFC-HOA plane wave driving function \cite[(5.1)]{Ahrens2012} (corrected here)
\begin{align}
D_{PW,HOA}(\mathbf{x}_0,\omega) = \sum\limits_{m=-\infty}^{+\infty} \underbrace{\frac{2 \mathrm{j}}{\frac{\omega}{c} r_0} \frac{(-\mathrm{j})^{|m|}}{h_{|m|}^{(2)}(\frac{\omega}{c} r_0)}
\mathrm{e}^{-\mathrm{j} m \phi_{PW}}}_{D_{PW}(m,\omega)}
\mathrm{e}^{\mathrm{j} m \phi_0}.
\end{align}
In \eqref{eq:DPWWFS} we can simplify and rearrange the dot products
\begin{align}
D_{PW,WFS}(\mathbf{x}_0,\omega)  =& -w(\mathbf{x}_0) \sqrt{\frac{\mathrm{j \omega}}{c}} \sqrt{8 \pi} \cdot
\sqrt{r_0} \cdot
[\cos\phi_0 \cos\phi_{PW} + \sin\phi_0 \sin\phi_{PW}]
\cdot
\mathrm{e}^{- \mathrm{j} \frac{\omega}{c}
[
r_0 \cos\phi_0 \cos\phi_{PW} + r_0 \sin\phi_0 \sin\phi_{PW}
]}\\
\label{eq:DPWWFS_FSPrep1}
=& -w(\mathbf{x}_0) \sqrt{\frac{\mathrm{j \omega}}{c}} \sqrt{8 \pi} \cdot
\sqrt{r_0} \cdot
[\cos(\phi_0-\phi_{PW})]
\cdot
\mathrm{e}^{- \mathrm{j} \frac{\omega}{c} r_0
\cos(\phi_0-\phi_{PW})}\\
=& -\sqrt{8 \pi r_0 \frac{\mathrm{j \omega}}{c}} \cdot
w(\mathbf{x}_0) \cdot
\frac{1}{2}[\mathrm{e}^{+\mathrm{j}(\phi_0-\phi_{PW})}+\mathrm{e}^{-\mathrm{j}(\phi_0-\phi_{PW})}]
\cdot
\mathrm{e}^{- \mathrm{j} \frac{\omega}{c} r_0
\cos(\phi_0-\phi_{PW})}
\end{align}

\begin{align}
\label{eq:DPWWFS_FSPrep2}
D_{PW,WFS}(\mathbf{x}_0,\omega) = -\sqrt{8 \pi r_0 \frac{\mathrm{j \omega}}{c}} \cdot
w(\mathbf{x}_0) \cdot
\frac{1}{2}[
\mathrm{e}^{+\mathrm{j}[(\phi_0-\phi_{PW})-\frac{\omega}{c} r_0
\cos(\phi_0-\phi_{PW})]}
+\mathrm{e}^{-\mathrm{j}[(\phi_0-\phi_{PW})+\frac{\omega}{c} r_0
\cos(\phi_0-\phi_{PW})]}
]
\end{align}
For the Fourier series $D(m) = \int_0^{2\pi} D(\phi_0) \mathrm{e}^{-\mathrm{j} m \phi_0}\mathrm{d}\phi_0$, we realize the multiplication of two $\phi_0$-dependent functions, i.e. $w(\mathbf{x}_0)$ and one of the exp-terms. If we would know the Fourier series of the individual functions, we could derive the final results by convolution of the Fourier series in $m$-domain. This sounds like a nice idea, it is however very unlikely that we can derive the convolution result as analytic equation.

Anyway, the Fourier series of the first exp-function in \eqref{eq:DPWWFS_FSPrep2} would be
\begin{align}
\frac{1}{2\pi}\int_0^{2\pi}
\mathrm{e}^{+\mathrm{j}[(\phi_0-\phi_{PW})-\frac{\omega}{c} r_0
\cos(\phi_0-\phi_{PW})]}
\e^{- \im m \phi_0 } \fsd \phi_0
\end{align}
With $\theta=\phi_0-\phi_{PW}$, $\fsd \theta / \fsd \phi_0 = 1$ we can write
\begin{calc}
\ExCalcCol{
\begin{align}
\frac{1}{2\pi}\int_0^{2\pi}
\mathrm{e}^{+\mathrm{j}[\theta-\frac{\omega}{c} r_0
\cos(\theta)]}
\e^{- \im m (\theta+\phi_{PW})} \fsd \theta
\end{align}
\begin{align}
\frac{1}{2\pi} \e^{- \im m \phi_{PW}} 
\int_0^{2\pi}
\mathrm{e}^{+\mathrm{j}[\theta-\frac{\omega}{c} r_0
\cos\theta]}
\e^{- \im m \theta} \fsd \theta
\end{align}
\begin{align}
\frac{1}{2\pi}
\e^{- \im m \phi_{PW}} 
\int_0^{2\pi}
\mathrm{e}^{+\mathrm{j}[(1-m)\theta-\frac{\omega}{c} r_0
\cos\theta]}
\fsd \theta
\end{align}
}
\end{calc}
\begin{align}
\frac{1}{2\pi}
\e^{- \im m \phi_{PW}} 
\int_0^{2\pi}
\mathrm{e}^{-\mathrm{j}[(m-1)\theta+\frac{\omega}{c} r_0
\cos\theta]}
\fsd \theta
\end{align}
Similarly, the Fourier series of the second exp-function in \eqref{eq:DPWWFS_FSPrep2} would be
\begin{align}
\frac{1}{2\pi}
\int_0^{2\pi}
\mathrm{e}^{-\mathrm{j}[(\phi_0-\phi_{PW})+\frac{\omega}{c} r_0
\cos(\phi_0-\phi_{PW})]}
\e^{- \im m \phi_0 } \fsd \phi_0
\end{align}
\begin{calc}
\ExCalcCol{
\begin{align}
\frac{1}{2\pi}
\int_0^{2\pi}
\mathrm{e}^{-\mathrm{j}[\theta+\frac{\omega}{c} r_0
\cos\theta]}
\e^{- \im m (\theta+\phi_{PW})} \fsd \theta
\end{align}
\begin{align}
\frac{1}{2\pi}
\e^{- \im m \phi_{PW}}
\int_0^{2\pi}
\mathrm{e}^{-\mathrm{j}[\theta+\frac{\omega}{c} r_0
\cos\theta]}
\e^{- \im m \theta} \fsd \theta
\end{align}
}
\end{calc}
\begin{align}
\frac{1}{2\pi}
\e^{- \im m \phi_{PW}}
\int_0^{2\pi}
\mathrm{e}^{-\mathrm{j}[(m+1)\theta+\frac{\omega}{c} r_0
\cos\theta]}
\fsd \theta
\end{align}
Both integrals combined with the factor $1/2$ yield the Fourier series analysis for the plane wave specific part
\begin{align}
\frac{1}{2\pi}
\frac{1}{2}
\e^{- \im m \phi_{PW}}
[
\int_0^{2\pi} \mathrm{e}^{-\mathrm{j}[(m-1)\theta+\frac{\omega}{c} r_0 \cos\theta]} \fsd \theta+
\int_0^{2\pi} \mathrm{e}^{-\mathrm{j}[(m+1)\theta+\frac{\omega}{c} r_0 \cos\theta]} \fsd \theta
]
\end{align}
We now introduce 
% $a=m-1$, $b=m+1$ and $z=\frac{\omega}{c} r_0 $ 
% \begin{align}
% \frac{1}{2\pi}
% \frac{1}{2}
% \e^{- \im m \phi_{PW}}
% [
% \int_0^{2\pi} \mathrm{e}^{-\mathrm{j}[a \theta + z \cos\theta]} \fsd \theta +
% \int_0^{2\pi} \mathrm{e}^{-\mathrm{j}[b \theta + z \cos\theta]} \fsd \theta
% ]
% \end{align}
% respectively with
$a=-(m-1)$, $b=-(m+1)$ and $z=-\frac{\omega}{c} r_0 $ 
\begin{align}
\frac{1}{2\pi}
\frac{1}{2}
\e^{- \im m \phi_{PW}}
[
\int_0^{2\pi} \mathrm{e}^{+\mathrm{j}[a \theta + z \cos\theta]} \fsd \theta +
\int_0^{2\pi} \mathrm{e}^{+\mathrm{j}[b \theta + z \cos\theta]} \fsd \theta
]
\end{align}
Expanding Euler yields
\begin{align}
\frac{1}{2\pi}
\frac{1}{2}
\e^{- \im m \phi_{PW}}
[
\int_0^{2\pi} (\cos(a\theta)+\im \sin(a\theta)) \mathrm{e}^{\mathrm{j} z \cos\theta} \fsd \theta+
\int_0^{2\pi} (\cos(b\theta)+\im \sin(b\theta)) \mathrm{e}^{\mathrm{j} z \cos\theta} \fsd \theta
]
\end{align}
For the first cosine integral we have
\begin{align}
\int_0^{2\pi} \cos(a\theta) \mathrm{e}^{\mathrm{j} z \cos\theta} \fsd \theta =
\int_0^{\pi} \cos(a\theta) \mathrm{e}^{\mathrm{j} z \cos\theta} \fsd \theta  + 
\int_\pi^{2\pi} \cos(a\theta) \mathrm{e}^{\mathrm{j} z \cos\theta} \fsd \theta 
\end{align}
which can be rearranged to
\begin{align}
\int_0^{2\pi} \cos(a\theta) \mathrm{e}^{\mathrm{j} z \cos\theta} \fsd \theta =
\int_0^{\pi} \cos(a\theta) \mathrm{e}^{\mathrm{j} z \cos\theta} \fsd \theta  + 
\int_0^{\pi} \cos(a(\theta+\pi)) \mathrm{e}^{\mathrm{j} z \cos(\theta+\pi)} \fsd \theta 
\end{align}
Then with NIST 10.9.2 (second integral $\theta^* = \theta + \pi$, subst, new int limits)
\begin{align}
\int_0^{2\pi} \cos(a\theta) \mathrm{e}^{\mathrm{j} z \cos\theta} \fsd \theta =
2 J_{a}(z) \frac{\pi}{\im^{-a}}
\end{align}
For the second cosine integral we have similarly
\begin{align}
\int_0^{2\pi} \cos(b\theta) \mathrm{e}^{\mathrm{j} z \cos\theta} \fsd \theta = 
\int_0^{\pi} \cos(b\theta) \mathrm{e}^{\mathrm{j} z \cos\theta} \fsd \theta +
\int_0^{\pi} \cos(b(\theta+\pi)) \mathrm{e}^{\mathrm{j} z \cos(\theta+\pi)} \fsd \theta
\end{align}
Again with NIST 10.9.2
\begin{align}
\int_0^{2\pi} \cos(b\theta) \mathrm{e}^{\mathrm{j} z \cos\theta} \fsd \theta = 
2 J_{b}(z) \frac{\pi}{\im^{-b}}
\end{align}
Combining both cosine integral results together with prefactor yields
\begin{align}
\frac{1}{2\pi}
\frac{1}{2}
\e^{- \im m \phi_{PW}}
(2 J_{a}(z) \frac{\pi}{\im^{-a}} + 2 J_{b}(z) \frac{\pi}{\im^{-b}})
=
\frac{1}{2}
\e^{- \im m \phi_{PW}}
(J_{a}(z) \frac{1}{\im^{-a}} + J_{b}(z) \frac{1}{\im^{-b}})
\end{align}
With $J_a(z) = (-1)^a J_a(-z)$ we get
\begin{align}
\frac{1}{2}
\e^{- \im m \phi_{PW}}
((-1)^a J_{a}(-z) \frac{1}{\im^{-a}} + (-1)^b J_{b}(-z) \frac{1}{\im^{-b}})
\end{align}
Re-substituting $a=-(m-1)$, $b=-(m+1)$ and $z=-\frac{\omega}{c} r_0 $
\begin{calc}
\ExCalcCol{
\begin{align}
\frac{1}{2}
\e^{- \im m \phi_{PW}}
((-1)^{-(m-1)} J_{-(m-1)}(\frac{\omega}{c} r_0) \frac{1}{\im^{m-1}} + (-1)^{-(m+1)} J_{-(m+1)}(\frac{\omega}{c} r_0) \frac{1}{\im^{m+1}})
\end{align}
}
\end{calc}
\begin{align}
\frac{1}{2 \im^m}
\e^{- \im m \phi_{PW}}
((-1)^{-(m-1)} J_{-(m-1)}(\frac{\omega}{c} r_0) \frac{1}{\im^{-1}} + (-1)^{-(m+1)} J_{-(m+1)}(\frac{\omega}{c} r_0) \frac{1}{\im^{+1}})
\end{align}
with $J_{-n} = (-1)^n J_n$
\begin{align}
\frac{1}{2 \im^m}
\e^{- \im m \phi_{PW}}
((-1)^{-(m-1)} (-1)^{+(m-1)} J_{m-1}(\frac{\omega}{c} r_0) \frac{1}{\im^{-1}} + (-1)^{-(m+1)} (-1)^{+(m+1)} J_{m+1}(\frac{\omega}{c} r_0) \frac{1}{\im^{+1}})
\end{align}
This yields to where we want to get (cf.  \cite[Table 1]{Hahn2016AES})
\begin{align}
\frac{\e^{- \im m \phi_{PW}}}{2\,\im^{m-1}}
(J_{m-1}(\frac{\omega}{c} r_0) - J_{m+1}(\frac{\omega}{c} r_0))
\end{align}
Altough not strictly proven yet, but numerically observed, we can assume that the sine-integrals have zero contribution.

The Fourier analysis of the plane wave part in \eqref{eq:DPWWFS_FSPrep1} is thus derived as
\begin{align}
\frac{1}{2 \pi}
\int\limits_0^{2 \pi}
\cos(\phi_0-\phi_{PW})
\, \mathrm{e}^{- \mathrm{j} \frac{\omega}{c} r_0
\cos(\phi_0-\phi_{PW})} \cdot \e^{- \im m \phi_0 } \fsd \phi_0 
= \frac{\e^{- \im m \phi_{PW}}}{2\,\im^{m-1}}
(J_{m-1}(\frac{\omega}{c} r_0) - J_{m+1}(\frac{\omega}{c} r_0))
\end{align}
and is generally complex-valued. For $\phi_{PW}=\pi/2$ and $\phi_{PW}=3\pi/2$  the series seems to be only imaginary.

The Fourier series of the spatial window $w(\vec{x}_0)$
\begin{equation}
w(\vec{x}_0) =
    \begin{cases}
      1 & \text{if}\qquad
       \unitn(\vec{x}_0) \cdot \hat{\vec{k}}_{PW}(\vec{x}_0) \geq 0 \\
      0 & \text{otherwise}
    \end{cases}.
\end{equation}
can be written as
\begin{align}
\frac{1}{2 \pi}& \int\limits_{\phi_{PW}+\pi/2}^{\phi_{PW}+3\pi/2} \e^{- \im m \phi_0 } \fsd \phi_0 =
\frac{1}{2 \pi} \frac{1}{-\im m} \e^{- \im m \phi_0 }\bigg|_{\phi_{PW}+\pi/2}^{\phi_{PW}+3\pi/2} =\\
\frac{1}{2 \pi}& \frac{1}{-\im m} (\e^{- \im m (\phi_{PW}+3\pi/2)) } - \e^{- \im m (\phi_{PW}+\pi/2))})=
\frac{1}{2 \pi} \frac{\im}{m} (\e^{- \im m (\phi_{PW}+3\pi/2)) } - \e^{- \im m (\phi_{PW}+\pi/2))})=\\
\frac{1}{2 \pi}& \frac{-\im}{m} (\e^{- \im m (\phi_{PW}+\pi/2))}-\e^{- \im m (\phi_{PW}+3\pi/2)) }),
\end{align}
thus the Fourier analysis of the spatial truncation window is
\begin{align}
\frac{1}{2 \pi}
\int\limits_0^{2 \pi} w(\vec{x}_0) \cdot \e^{- \im m \phi_0 } \fsd \phi_0 = \frac{1}{2 \pi}& \frac{-\im}{m} (\e^{- \im m (\phi_{PW}+\pi/2))}-\e^{- \im m (\phi_{PW}+3\pi/2)) })
\end{align}
and in general complex valued. For the special case $\phi_{PW} = \pi$ we obtain
\begin{align}
\frac{1}{2 \pi} \frac{-\im}{m} (\e^{- \im m \frac{3 \pi}{2}}-\e^{- \im m \frac{5 \pi}{2} }) =
\frac{1}{2 \pi} \frac{-\im}{m} (\e^{+ \im m \frac{\pi}{2}}-\e^{- \im m \frac{\pi}{2} })
\end{align}
and further a real-valued sinc()-function characteristics
\begin{align}
\frac{1}{\pi} \frac{-\im\im}{m} \frac{\e^{+ \im m \frac{\pi}{2}}-\e^{- \im m \frac{\pi}{2} }}{2 \im} = \frac{1}{2} \frac{\sin(m\frac{\pi}{2})}{m\frac{\pi}{2}}
\end{align}
For $\phi_{PW} = 0$, we similarly obtain $-\frac{1}{2} \frac{\sin(m\frac{\pi}{2})}{m\frac{\pi}{2}}$.

%The complex-valued convolution of both Fourier series is thus 
%\begin{align}
%\left[\frac{\e^{- \im m \phi_{PW}}}{2\,\im^{m-1}}
%(J_{m-1}(\frac{\omega}{c} r_0) - J_{m+1}(\frac{\omega}{c} r_0))\right]
%*_m
%\left[\frac{1}{2 \pi} \frac{-\im}{m} (\e^{- \im m (\phi_{PW}+\pi/2))}-\e^{- \im m (\phi_{PW}+3\pi/2)) })\right],
%\end{align}
%doubtful to find an analytic expression for the result.

The complex-valued convolution of both Fourier series together with the prefactors, i.e. the final Fourier series result, is thus 
\begin{align}
D_{PW,WFS}(m,\omega) = 
-\sqrt{\frac{\mathrm{j \omega}}{c}} \sqrt{8 \pi r_0} \cdot
&\left[\frac{\mathrm{e}^{- \mathrm{j} m \phi_{PW}}}{2\,\mathrm{j}^{m-1}}
(J_{m-1}(\frac{\omega}{c} r_0) - J_{m+1}(\frac{\omega}{c} r_0))\right]
*_m \\
&\left[\frac{1}{2 \pi} \frac{-\mathrm{i}}{m} (\mathrm{e}^{- \mathrm{i} m (\phi_{PW}+\pi/2))}-\mathrm{e}^{- \mathrm{i} m (\phi_{PW}+3\pi/2)) })\right]
\end{align}
doubtful to find an analytic expression for the result.


\section{Acknowledgment}
We thank Naha Hahn (github: narahahn) for fruitful discussions and helpful comments on the calculus.

\bibliographystyle{IEEEtranSA.bst}
\bibliography{literature}
\end{document}





































